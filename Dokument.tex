\documentclass[a4]{article}

\usepackage{amsfonts}
\usepackage{amsmath}
\usepackage{amssymb}
\usepackage[ngerman]{babel}
\usepackage{graphicx}
\usepackage{hyperref}

\title{Wissenschaftliche Texte schreiben mit \LaTeX}
\author{Hendrik Sauer}

\begin{document}

\maketitle

We're no \emph{strangers} to \textbf{love}.
You know the rules and so do I.

\section{A full commitment's}\label{sec:first-verse}

what I'm thinking of.

\subsection{You wouldn't get this}

from any other guy.

\section*{I just wanna tell you}

how I'm feeling.
Gotta make you understand.

Never gonna
\begin{itemize}
    \item Give you up
    \item Let you down
    \begin{itemize}
        \item run around
        \item and desert you
    \end{itemize}
\end{itemize}

Never gonna
\begin{enumerate}
    \item Make you cry
    \item say goodbye
    \begin{enumerate}
        \item Tell a lie
        \item and hurt you
    \end{enumerate}
\end{enumerate}

\begin{figure}
    \centering
    \includegraphics[width=0.7\textwidth]{rick-astley.jpg}
    \caption{Rick Astley}
    \label{fig:rick-astley}
\end{figure}

\section{Quick Math}

Seien \(x,y \in \mathbb{R}\). Es gilt
\[ |x+y| \leq |x| + |y|. \]
Sei \(z = (z_1, \dots, z_n) \in \mathbb{R}^n\)
\begin{equation}
    ||z||_p = \left(\sum_{i=1}^n z_i^p\right)^{\frac{1}{p}} \label{eq:p-norm}
\end{equation}

\section{Kreuzverweise}

Wie in \autoref{sec:first-verse} zu lesen ist, in \autoref{fig:rick-astley} zu sehen und in \autoref{eq:p-norm} gezeigt, ist \LaTeX{} 1 nices Tool zur Texterstellung.

\end{document}
